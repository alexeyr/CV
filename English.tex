\documentclass[12pt,a4paper,sans]{moderncv}        % possible options include font size ('10pt', '11pt' and '12pt'), paper size ('a4paper', 'letterpaper', 'a5paper', 'legalpaper', 'executivepaper' and 'landscape') and font family ('sans' and 'roman')

% moderncv themes
\moderncvstyle{casual}                             % style options are 'casual' (default), 'classic', 'banking', 'oldstyle' and 'fancy'
\moderncvcolor{blue}                               % color options 'black', 'blue' (default), 'burgundy', 'green', 'grey', 'orange', 'purple' and 'red'
%\renewcommand{\familydefault}{\sfdefault}         % to set the default font; use '\sfdefault' for the default sans serif font, '\rmdefault' for the default roman one, or any tex font name
%\nopagenumbers{}                                  % uncomment to suppress automatic page numbering for CVs longer than one page

% character encoding
%\usepackage[utf8]{inputenc}                       % if you are not using xelatex ou lualatex, replace by the encoding you are using
%\usepackage{CJKutf8}                              % if you need to use CJK to typeset your resume in Chinese, Japanese or Korean

% adjust the page margins
%\usepackage[scale=0.75]{geometry}
%\setlength{\hintscolumnwidth}{3cm}                % if you want to change the width of the column with the dates
%\setlength{\makecvtitlenamewidth}{10cm}           % for the 'classic' style, if you want to force the width allocated to your name and avoid line breaks. be careful though, the length is normally calculated to avoid any overlap with your personal info; use this at your own typographical risks...

% personal data
\name{Alexey}{Romanov}
\address{Zelenograd}{Moscow}{Russia}               % optional, remove / comment the line if not wanted; the "postcode city" and "country" arguments can be omitted or provided empty
%\phone[mobile]{+7~(234)~567~890}
\email{alexey.v.romanov@gmail.com}
\social[linkedin]{alexeyromanov}                        % optional, remove / comment the line if not wanted
\social[twitter]{alexey_r}
\social[github]{alexeyr}
\extrainfo{%
\httplink[stackoverflow]{https://stackoverflow.com/users/9204/alexey-romanov}%
}
%\photo[64pt][0.4pt]{picture}                       % optional, remove / comment the line if not wanted; '64pt' is the height the picture must be resized to, 0.4pt is the thickness of the frame around it (put it to 0pt for no frame) and 'picture' is the name of the picture file

% bibliography adjustements (only useful if you make citations in your resume, or print a list of publications using BibTeX)
%   to show numerical labels in the bibliography (default is to show no labels)
\makeatletter\renewcommand*{\bibliographyitemlabel}{\@biblabel{\arabic{enumiv}}}\makeatother
%   to redefine the bibliography heading string ("Publications")
%\renewcommand{\refname}{Articles}

% bibliography with mutiple entries
%\usepackage{multibib}
%\newcites{book,misc}{{Books},{Others}}
%----------------------------------------------------------------------------------
%            content
%----------------------------------------------------------------------------------
\begin{document}
%-----       resume       ---------------------------------------------------------
\makecvtitle

\section{Education}
\cventry{1998--2003}{Specialist in Applied Mathematics}{Moscow State University}{Moscow}{Diploma with honours (Red diploma)}{Specialization: Mathematics, Mathematical Logic}

% \section{Master thesis}
% \cvitem{title}{\emph{Title}}
% \cvitem{supervisors}{Shehtman~B.V.}
% \cvitem{description}{Short thesis abstract}

\section{Professional Skills}
\subsection{Primary}
\cvlistitem{Scala}
\cvlistitem{Kotlin}
\cvlistitem{Java}
\cvlistitem{Erlang}
\cvlistitem{Algorithms and data structures}
\cvlistitem{Concurrent and parallel programming}
\cvlistitem{Teaching, Mentoring}
\subsection{Intermediate}
\cvlistitem{Languages: SQL, C\#, Haskell, JavaScript, Prolog}
\cvlistitem{Databases (particularly SQLite)}
\cvlistitem{Compilers, optimizations, code transformation}
\cvlistitem{Eclipse RCP}
\cvlistitem{IDEA/Android Studio plugin development}
\subsection{Basic}
\cvlistitem{Languages: C, C++, Rust, OCaml, Lua, Scheme/Racket}
\cvlistitem{Android development}
\cvlistitem{Machine Learning}
% \cvdoubleitem{category 2}{XXX, YYY, ZZZ}{category 5}{XXX, YYY, ZZZ}
% \cvdoubleitem{category 3}{XXX, YYY, ZZZ}{category 6}{XXX, YYY, ZZZ}

\section{Main Projects}

\cventry{2014--present}{\textbf{Scalan} (Huawei Research): Domain-specific optimization and specialization framework}
\begin{itemize}
  \item Skills: Scala, Compilers, Optimization, Code transformation
  \item Achievements: One of main developers of Scalan framework and Scala compiler plugin for this framework
  \item Effects: coauthor of a WGP2014 paper, 2 patents filed
\end{itemize}

\cventry{2016}{\textbf{SQLite Kernel Engine} (Huawei Research): }
\begin{itemize}
  \item Skills: Scala, C++, Lua, Compilers, Optimization, Code transformation
  \item Achievements: Lua and C++ code generation from SQL queries, SQLite calls generated code instead of its own bytecode interpreter when detecting a repeated query
  \item Effects: 1.5 to 10x speedup on some queries, patent filed
\end{itemize}

\cventry{2010--2014}{\textbf{erlang-sqlite3} (Focus-Media, then side-project): Erlang library for work with SQLite3 databases}
\begin{itemize}
  \item Skills: Erlang, C, SQLite, Optimization
  \item Achievements: Took over maintenance and development of an abandoned Erlang library for working with SQLite3 databases.
  Attracted users and contributors.
  \item Effects: used in projects inside Focus-Media and by external developers
\end{itemize}

\cventry{2010--2012}{A system for automatically downloading and playing video on remote devices (Focus-Media) }
\begin{itemize}
  \item Skills: Erlang, C, Java, Network programming, Concurrent programming, Optimization
  \item Achievements: Responsible for developing the client side scheduling and processing, assisted with server side
  \item Effects: stable processing, successful deployment to multiple satisfied customers
\end{itemize}

\cventry{2011--2012}{Eclipse RCP projects (Focus-Media)}
\begin{itemize}
  \item Skills: Scala, Java, OSGi, Eclipse RCP, Network programming, UI design
  \item Achievements: Developer on a SCADA system, responsible for aggregating sensor data timeseries, allowing and processing user queries on the data. Developer on automated workplace in a security system.
\end{itemize}

\cventry{2013}{Automated phone customer service system (LikeMelody LLC)}
\begin{itemize}
  \item Skills: Natural Language Processing, Machine Learning, Artificial Intelligence
  \item Achievements: Architect and developer on a system automating first-line phone call center for small business (unreleased when I left the company).
\end{itemize}

\cventry{2004--present}{Teaching (Moscow Institute of Electronic Technology)}
\begin{itemize}
  \item Skills: Mentoring, Teaching, Haskell, Erlang, Prolog
  \item Achievements: Developed and taught new courses: Mathematical Logic and Algorithm Theory, Functional Programming (in Haskell), Functional and Logical Programming (in Erlang and Prolog); Supervised master students
\end{itemize}
% \cventry{2010--2012}{Developer}{Focus-Media}{Zelenograd}{}{%General description no longer than 1--2 lines.\newline{}%
% Achievements:%
% \begin{itemize}%
% \item Achievement 1;
% \item Achievement 2, with sub-achievements:
%   \begin{itemize}%
%   \item Sub-achievement (a);
%   \item Sub-achievement (b), with sub-sub-achievements (don't do this!);
%   \item Sub-achievement (c);
%   \end{itemize}
% \item Achievement 3.
% \end{itemize}}

% \cventry{year--year}{Job title}{Employer}{City}{}{Description line 1\newline{}Description line 2}

\section{Languages}
\cvitem{Russian}{Native}
\cvitem{English}{Fluent (written), Advanced (spoken)}

% \section{Interests}
% \cvitem{hobby 1}{Description}
% \cvitem{hobby 2}{Description}
% \cvitem{hobby 3}{Description}

% \section{Extra 1}
% \cvlistitem{Item 1}
% \cvlistitem{Item 2}
% \cvlistitem{Item 3. This item is particularly long and therefore normally spans over several lines. Did you notice the indentation when the line wraps?}

% \section{Extra 2}
% \cvlistdoubleitem{Item 1}{Item 4}
% \cvlistdoubleitem{Item 2}{Item 5\cite{book1}}
% \cvlistdoubleitem{Item 3}{Item 6. Like item 3 in the single column list before, this item is particularly long to wrap over several lines.}

% \section{References}
% \begin{cvcolumns}
%   \cvcolumn{Category 1}{\begin{itemize}\item Person 1\item Person 2\item Person 3\end{itemize}}
%   \cvcolumn{Category 2}{Amongst others:\begin{itemize}\item Person 1, and\item Person 2\end{itemize}(more upon request)}
%   \cvcolumn[0.5]{All the rest \& some more}{\textit{That} person, and \textbf{those} also (all available upon request).}
% \end{cvcolumns}

% Publications from a BibTeX file without multibib
%  for numerical labels: \renewcommand{\bibliographyitemlabel}{\@biblabel{\arabic{enumiv}}}% CONSIDER MERGING WITH PREAMBLE PART
%  to redefine the heading string ("Publications"): \renewcommand{\refname}{Articles}
\nocite{*}
\bibliographystyle{plain}
\bibliography{publications}                        % 'publications' is the name of a BibTeX file

% Publications from a BibTeX file using the multibib package
%\section{Publications}
%\nocitebook{book1,book2}
%\bibliographystylebook{plain}
%\bibliographybook{publications}                   % 'publications' is the name of a BibTeX file
%\nocitemisc{misc1,misc2,misc3}
%\bibliographystylemisc{plain}
%\bibliographymisc{publications}                   % 'publications' is the name of a BibTeX file

\clearpage
% %-----       letter       ---------------------------------------------------------
% % recipient data
% \recipient{Company Recruitment team}{Company, Inc.\\123 somestreet\\some city}
% \date{January 01, 1984}
% \opening{Dear Sir or Madam,}
% \closing{Yours faithfully,}
% \enclosure[Attached]{curriculum vit\ae{}}          % use an optional argument to use a string other than "Enclosure", or redefine \enclname
% \makelettertitle

% Lorem ipsum dolor sit amet, consectetur adipiscing elit. Duis ullamcorper neque sit amet lectus facilisis sed luctus nisl iaculis. Vivamus at neque arcu, sed tempor quam. Curabitur pharetra tincidunt tincidunt. Morbi volutpat feugiat mauris, quis tempor neque vehicula volutpat. Duis tristique justo vel massa fermentum accumsan. Mauris ante elit, feugiat vestibulum tempor eget, eleifend ac ipsum. Donec scelerisque lobortis ipsum eu vestibulum. Pellentesque vel massa at felis accumsan rhoncus.

% Suspendisse commodo, massa eu congue tincidunt, elit mauris pellentesque orci, cursus tempor odio nisl euismod augue. Aliquam adipiscing nibh ut odio sodales et pulvinar tortor laoreet. Mauris a accumsan ligula. Class aptent taciti sociosqu ad litora torquent per conubia nostra, per inceptos himenaeos. Suspendisse vulputate sem vehicula ipsum varius nec tempus dui dapibus. Phasellus et est urna, ut auctor erat. Sed tincidunt odio id odio aliquam mattis. Donec sapien nulla, feugiat eget adipiscing sit amet, lacinia ut dolor. Phasellus tincidunt, leo a fringilla consectetur, felis diam aliquam urna, vitae aliquet lectus orci nec velit. Vivamus dapibus varius blandit.

% Duis sit amet magna ante, at sodales diam. Aenean consectetur porta risus et sagittis. Ut interdum, enim varius pellentesque tincidunt, magna libero sodales tortor, ut fermentum nunc metus a ante. Vivamus odio leo, tincidunt eu luctus ut, sollicitudin sit amet metus. Nunc sed orci lectus. Ut sodales magna sed velit volutpat sit amet pulvinar diam venenatis.

% Albert Einstein discovered that $e=mc^2$ in 1905.

% \[ e=\lim_{n \to \infty} \left(1+\frac{1}{n}\right)^n \]

% \makeletterclosing
\end{document}